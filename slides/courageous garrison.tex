\frame{
\frametitle{Puzzle time - on adversarial search}

\begin{block}{The courageous garrison}
A courageous garrison was defending a snow fort. The commander arranged his
forces as shown in the square frame (the inner square showing the garrison's total
strength of 40 boys): 11 boys defending each side of the fort. 
The garrison ``lost'' 4 boys during each of the first, second, third, and fourth
assaults, and 2 during the fifth and last. But after each charge 11 boys defended
each side of the snow fort. How?
(puzzle 101 from Kordemsky. The Moscow Puzzles - 359 mathematical recreations)
\end{block}
\begin{columns}
 \begin{column}{0.65\textwidth}
 \includegraphics[width=6.3cm]{fig/garrisson.png}   
 \end{column}

 \begin{column}{0.37\textwidth}
 \includegraphics[width=3cm]{fig/fort.png}  
 
 \only<2->{1st attack: 36 brave defenders}
 
 \only<3->{2nd attack: 32 brave defenders}
 
 \only<4->{3rd attack: 28 brave defenders}
 
 \only<5->{4th attack: 24 brave defenders}
 
 \only<6->{5th attack: 22 brave defenders}
 %\only<7->{Day 5: 22 brave defenders (other solution)}
 \end{column}

\end{columns}

\begin{center}

 
\end{center}

}

