\begin{frame}[plain]{Constraint model for N-queens}
\begin{columns}
\begin{column}{0.57\textwidth}
Place n-queens on an $n \times n$ board so that no pair of queens attacks each other

\begin{description}
\item[Variables:] \only<2->{$x_1, x_2, x_3, x_4$}
\item[Domains:]  \only<3->{$\{1,2,3,4\}$}
\item[Constraints:] 
\only<4->{$x_i  \neq x_j$ \\
   $| x_i - x_j | \neq | i - j |$} 

\item[A solution:]
\only<5->{$x_1 \leftarrow 3$, $x_2 \leftarrow 1$, $x_3 \leftarrow 4$, $x_4 \leftarrow 2$}
\end{description}
\end{column}

\begin{column}{0.43\textwidth}
\pgfimage[width=5.5cm]{fig/robot2.png} \\
 
\end{column}
\end{columns}
\begin{example}[Fancy queens]
I have placed a queen in one of the white squares of the board shown. Place 7 more
queens in white squares so that no 2 of the 8 queens are in line horizontally, vertically,
or diagonally
\end{example}
\pgfimage[width=3cm]{fig/diag.png} 
\hfill
\pgfimage[width=3cm]{fig/diag2.png} 
\end{frame}

\frame[plain]{
\centering
 \pgfimage[width=10cm]{fig/fancy.png} 

 \begin{itemize}
\item \mat{$Q(x,y)$} is a predicate, not a function 
 \item Note the usage of \texttt{domain\_size}
 \item Note the usage of modules
 \item Note the usage of the minus operator
\end{itemize}
}

