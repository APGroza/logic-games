\frame{
\frametitle{Puzzle time - on propositional logic}

\begin{block}{Ladies and tigers: the story}
I welcome you here to the world of ladies and tigers built by Smullyan
upon the short story published by F. R. Stockton, The \textit{Lady, or the Tiger?}, 25th century 83, 83-86 (1882). 
You are a young prince in love with a beautiful princess. 
Her father, the ruler of the kingdom, asked you to choose between two doors, behind each there is either your beloved one or a killing tiger.
There are signs on each door, but you don't know if they are true or false. 
Your dilemma is: which door to open in order to find the lady, marry her, and get half of the kingdom (as compensation, the wagging tongues would say).
\end{block}
\centering
\includegraphics[width=0.5\textwidth]{fig/lady.jpeg} 

}

\frame{
\begin{block}{First task: Each of the two rooms contains either a lady or a tiger.}
 There could be tigers in both
rooms or ladies in both rooms. There is a sign on each room. One of the signs is true and
the other is false. Which door to open in order to find the lady, marry her, and get half of
the kingdom as compensation? 
\end{block}

\centering
\includegraphics[width=0.75\textwidth]{fig/l1.png} 
\vfill
}

\begin{frame}[fragile]
\begin{block}{First task}
Each of the two rooms contains either a lady or a tiger. There could be tigers in both
rooms or ladies in both rooms. There is a sign on each room. One of the signs is true and
the other is false. Which door to open in order to find the lady, marry her, and get half of
the kingdom as compensation? 
\end{block}

\begin{center}
\includegraphics[width=0.8\textwidth]{fig/l1.png}  
\end{center}

\begin{columns}
\begin{column}{0.55\textwidth}
\begin{lstlisting}
assign(max_models,-1).
assign(domain_size,2).

formulas(assumptions).
  L1 & L2 | L1 & T2 | L2 & T1 | T1 & T2.
  (L1 -> -T1) & (L2 -> -T2).
  R1 <-> (L1 & T2).
  R2 <-> ((L1 | L2) & (T1 | T2)).
  (R1 | R2) & -(R1 & R2).
end_of_list.
\end{lstlisting}
 \end{column}

%\only<3->{
\begin{column}{0.5\textwidth}
\begin{tabular}{c|cccccc}
 Model   & $L_1$ & $L_2$ & $T_1$ & $T_2$ & $R_1$ & $R_2$ \\
  \hline
  1 & 0 & 1 & 1 & 0 & 0 & 1\\
  \end{tabular}
\includegraphics[width=\textwidth]{fig/l1sol.png} 
\end{column}
%}
 
\end{columns}


\end{frame}

