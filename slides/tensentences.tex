\frame{\frametitle{}
\arenabox{Ten sentences}{green!40}{green!10}{
In a list of 10 statements, the $n^{th}$ statement says there are exactly $n$ false items in the list. 
Which of the statements, if any, are false?
\begin{enumerate}
 \item In this list exactly 1 statement is false.
 \item In this list exactly 2 statements are false.
 \item In this list exactly 3 statements are false.
 \item In this list exactly 4 statements are false.
 \item In this list exactly 5 statements are false.
 \item In this list exactly 6 statements are false.
 \item In this list exactly 7 statements are false.
 \item In this list exactly 8 statements are false.
 \item In this list exactly 9 statements are false.
 \item In this list exactly 10 statements are false.
\end{enumerate}
}
}


\frame[plain]{
\begin{itemize}
 \item Function $m(x)$ stating if the $x^{th}$ message is true (\mat{$m(x) = 1$}) or false (\mat{$m(x) = 0)$}
 \item Each $x^{th}$ statement says there are exactly $x$ false items in the list:
\begin{center}
 \mat{$m(x) = 1 \leftrightarrow NoFalse = x$}
\end{center}
\item Since \mat{$m(x)$} is a function, we can use it in computations
\end{itemize}

\begin{center}
\includegraphics[width=8cm]{fig/ten.png} 
\end{center}

\begin{itemize}
\item Mace4 finds a single model: 
\end{itemize}

\begin{center}
 \mat{\begin{tabular}{llll}
 function(NoFalse,& 9),& function(NoTrue, & [1]),\\
 function(NoSentences, & [10]), & function(m(\_),  & [\alert{0}, 0, 0, 0, 0, 0, 0, 0, 0, 1, 0 ])
 \end{tabular}}
\end{center}

\arenabox{Ten sentences relaxed}{green!40}{green!10}{
In a list of 10 statements, the $n^{th}$ statement says there are at least $n$ false items in the list. 
Which of the statements, if any, are false?
}
}

